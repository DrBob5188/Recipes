%%Chapter - Split into separate file if too large
\chapter{Finger Food}
%%Start recipe
\newrecipe{Caramelised Onion Tarts}{}
\section*{Directions}
\begin{enumerate}
	\item Make the balsamic onion jam as directed in the \hyperlink{caramelised_onion}{recipe} on page \pageref{caramelised_onion}
	\item Spoon mixture into pre-made pastry cases and top with crumbled feta, goats cheese or blue cheese.
\end{enumerate}
%%End recipe

%%Start recipe
\newrecipe{Roast Beef Canapes}{http://www.taste.com.au/recipes/18552/roast_beef_canapes}
\section*{Ingredients}
\begin{ingredients-list}
	\item 30cm-long baguette bread
	\item olive oil cooking spray
	\item 200g sliced rare roast beef, cut into strips
	\item 200g roasted red capsicum, thinly sliced
	\item Horseradish mayonnaise
	\item \sfrac{1}{2} cup whole-egg mayonnaise
	\item 1 tablespoon horseradish cream
	\item 1\sfrac{1}{2} teaspoons Dijon mustard 
\end{ingredients-list}
\section*{Directions}
\begin{enumerate}
	\item Preheat oven to 180°C. Trim ends from bread. Cut into 24, 5mm-thick slices. Place bread slices, in a single layer, on 2 oven trays. 
	Spray with oil. Bake for 8 to 10 minutes, swapping trays over halfway through cooking, or until light golden. Remove to a wire rack to cool completely.
	\item Make horseradish mayonnaise: Combine mayonnaise, horseradish cream and mustard in a small bowl. Season with salt and pepper.
	\item Top each bread slice with roast beef. Dollop with horseradish mayonnaise and top with capsicum. Season with pepper. Serve immediately.
\end{enumerate}
%%End recipe

%%Start recipe
\newrecipe{Corn Fritters}{}
\section*{Ingredients}
\begin{ingredients-list}
	\item 420g can corn kernels, drained
	\item 1 cup or 150g self-raising flour
	\item 2 shallots, trimmed and sliced
	\item 2 eggs
	\item \sfrac{1}{4} cup or 60ml milk
	\item Butter, to grease
\end{ingredients-list}
\section*{Directions}
\begin{enumerate}
	\item To make the fritters, combine corn kernels, flour and shallot in a bowl. Make a well in the centre and add the eggs and milk.
		Stir gradually incorporating liquid with dry ingredients until combined.
	\item Lightly grease a frying pan with butter. Heat over a medium-low heat. Cook 2 corn fritters at a time - l/4 cup corn mixture makes 1 fritter.
		Cook for 2-3 minutes each side, or until golden.
	\item Set aside. Repeat with remaining mixture
\end{enumerate}
%%End recipe

%%Start recipe
\newrecipe{Chicken Satays}{http://www.taste.com.au/recipes/5925/chicken_satays}
\section*{Ingredients}
\begin{ingredients-list}
	\item 1 cup coconut cream
	\item 2 tbsp. good-quality mild curry powder
	\item 4 garlic cloves
	\item 2 tbsp. brown sugar
	\item 40ml (2 tbsp.) Thai fish sauce
	\item 4 tbsp. chopped coriander root and stem
	\item 1kg chicken breast fillets, cut into 2cm pieces 
\end{ingredients-list}
\section*{Directions}
\begin{enumerate}
	\item Place coconut cream, curry, garlic, sugar, fish sauce and coriander in a blender and process until smooth. Place chicken in a non-metallic bowl, add marinade and stir well. Cover and refrigerate overnight.
	\item The next day, thread 3-4 pieces of chicken on each skewer, and then grill until cooked through. Serve on platters with mango chutney, if desired. 
\end{enumerate}
%%End recipe
